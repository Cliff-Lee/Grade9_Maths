\documentclass[11pt]{article}
\usepackage[margin=1in]{geometry}
\usepackage{amsmath,amssymb}
\usepackage{array,booktabs}
\usepackage{multicol}
\usepackage{enumitem}
\usepackage{graphicx}
\usepackage{xcolor}
\setlist[itemize]{left=0pt,itemsep=2pt,topsep=4pt}
\setlist[enumerate]{left=0pt,itemsep=4pt,topsep=6pt}

\newcommand{\ruleline}{\par\noindent\rule{\textwidth}{0.4pt}\par}
\newcommand{\blank}[1]{\makebox[#1]{\hrulefill}}
\newcommand{\sq}{\(\square\)}

\begin{document}

\begin{center}
{\LARGE \textbf{Guided Investigation: Mirror \& Spin Lab — Even/Odd Functions}}\\[4pt]
\textit{Discover symmetry by exploring an interactive applet.}
\end{center}

\vspace{0.5em}
\noindent \textbf{Name:} \blank{2.5in} \hfill \textbf{Class:} \blank{1.5in} \hfill \textbf{Date:} \blank{1.0in}

\ruleline

\section*{Purpose \& Learning Goals}
\begin{itemize}
  \item Use the Mirror (y-axis reflection) and Spin (180$^\circ$ about the origin) overlays to \textbf{detect patterns} in graphs.
  \item \textbf{Form a conjecture} about when a function matches the Mirror overlay versus the Spin overlay.
  \item \textbf{Test, refine, and justify} your conjecture using examples and non-examples.
\end{itemize}

\section*{Materials}
\begin{itemize}
  \item Computer with the \textit{Mirror \& Spin Lab} applet open (no login or submission needed).
  \item This investigation sheet and a pen/pencil.
\end{itemize}

\section*{Ground Rules}
\begin{itemize}
  \item The goal is \textbf{discovery}. Avoid using outside definitions; rely on \textbf{what you observe}.
  \item Record \textbf{clear evidence} (screenshots not required). Use the tables below.
  \item If the applet shows an expression error, adjust your input (e.g., use \texttt{|x|} for absolute value; \texttt{ln(x)} for natural log; \verb|^| for powers).
\end{itemize}

\ruleline

\section*{Part A — Orientation (5--8 min)}
\begin{enumerate}
  \item In the applet, ensure \textbf{Mirror} is checked and \textbf{Spin} is unchecked. Try the presets \texttt{x\^{}2}, \texttt{|x|}, \texttt{cos(x)}.
  \begin{itemize}
    \item What happens when the Mirror overlay \emph{matches exactly} the blue graph? Describe in your own words: \\
    \vspace{1.4em}\ruleline
  \end{itemize}
  \item Now uncheck \textbf{Mirror} and check \textbf{Spin}. Try \texttt{x}, \texttt{x\^{}3}, \texttt{sin(x)}.
  \begin{itemize}
    \item What happens when the Spin overlay \emph{matches exactly}? Describe: \\
    \vspace{1.4em}\ruleline
  \end{itemize}
\end{enumerate}

\section*{Part B — Collect Evidence (10--12 min)}
Use the table to log at least \textbf{8 functions}: aim for at least 3 that match Mirror, 3 that match Spin, and 2 that match neither. Use the applet's classification panel only to \emph{check} your thinking.

\renewcommand{\arraystretch}{1.25}
\begin{tabular}{@{}p{0.30\linewidth}p{0.18\linewidth}p{0.42\linewidth}@{}}
\toprule
\textbf{Function \(f(x)\)} & \textbf{Overlay Match} & \textbf{What you noticed (shape, symmetry, values at \(\pm x\), etc.)} \\
\midrule
\blank{.28\linewidth} & Mirror / Spin / Neither & \blank{.40\linewidth} \\
\blank{.28\linewidth} & Mirror / Spin / Neither & \blank{.40\linewidth} \\
\blank{.28\linewidth} & Mirror / Spin / Neither & \blank{.40\linewidth} \\
\blank{.28\linewidth} & Mirror / Spin / Neither & \blank{.40\linewidth} \\
\blank{.28\linewidth} & Mirror / Spin / Neither & \blank{.40\linewidth} \\
\blank{.28\linewidth} & Mirror / Spin / Neither & \blank{.40\linewidth} \\
\blank{.28\linewidth} & Mirror / Spin / Neither & \blank{.40\linewidth} \\
\blank{.28\linewidth} & Mirror / Spin / Neither & \blank{.40\linewidth} \\
\bottomrule
\end{tabular}

\section*{Part C — Make a Conjecture (6--8 min)}
\begin{enumerate}
  \item Based on your table, write a \textbf{first draft} rule for when a function matches the Mirror overlay: \\[0.6em]
  \ruleline
  \item Write a \textbf{first draft} rule for when a function matches the Spin overlay: \\[0.6em]
  \ruleline
  \item Without calculating, what \emph{relationship} might hold between the y-values at \(x\) and \(-x\) when the Mirror overlay matches? When the Spin overlay matches?\\[0.6em]
  \ruleline
\end{enumerate}

\section*{Part D — Stress Test Your Conjecture (10--15 min)}
Use the applet's \textbf{Transform} chips to modify functions and see what happens.

\subsection*{D1. Vertical/Horizontal Shifts}
\begin{itemize}
  \item Start with a function that matched Mirror. Try \(f(x)+c\) and \(f(x-h)\). Which (if any) changes kept the match? Which broke it? Why might that be?
\end{itemize}
\noindent Observations: \\
\ruleline

\subsection*{D2. Multiply by \(-1\) and by \(x\)}
\begin{itemize}
  \item Pick a function that matched Mirror. Try \(-f(x)\) and \(x\cdot f(x)\). What happens to the overlay match?
  \item Repeat with a function that matched Spin.
\end{itemize}
\noindent Observations: \\
\ruleline

\subsection*{D3. Stretching}
\begin{itemize}
  \item Try \(k\cdot f(x)\) and \(f(kx)\) with \(k>0\). Which transformations preserve each kind of match?
\end{itemize}
\noindent Observations: \\
\ruleline

\subsection*{D4. Edge Cases}
\begin{itemize}
  \item Explore \(\,0\) (the zero function), \(|x|\), \(\ln(x)\), \(\sqrt{x}\). Note any domain issues and how they affect matching.
\end{itemize}
\noindent Observations: \\
\ruleline

\section*{Part E — Refine \& Justify (8--10 min)}
\begin{enumerate}
  \item \textbf{Refined Conjecture (Mirror):}\\
    \ruleline
  \item \textbf{Refined Conjecture (Spin):}\\
    \ruleline
  \item \textbf{Evidence set.} List two functions that \emph{support} each conjecture and one \emph{counterexample} for each (explain why it fails).\\[0.3em]
  \begin{tabular}{@{}p{0.32\linewidth}p{0.28\linewidth}p{0.34\linewidth}@{}}
  \toprule
  \textbf{Conjecture} & \textbf{Example/Counterexample} & \textbf{Why this supports or refutes} \\
  \midrule
  Mirror & Example: \blank{.24\linewidth} & \blank{.30\linewidth} \\
  Mirror & Example: \blank{.24\linewidth} & \blank{.30\linewidth} \\
  Mirror & Counterexample: \blank{.18\linewidth} & \blank{.30\linewidth} \\
  \midrule
  Spin & Example: \blank{.26\linewidth} & \blank{.30\linewidth} \\
  Spin & Example: \blank{.26\linewidth} & \blank{.30\linewidth} \\
  Spin & Counterexample: \blank{.20\linewidth} & \blank{.30\linewidth} \\
  \bottomrule
  \end{tabular}
\end{enumerate}

\section*{Numeric Spot-Checks (optional)}
Pick 3 values of \(x\) and compare \(f(x)\), \(f(-x)\), and \(-f(x)\) for a function of your choice.
\[
\begin{array}{r|c|c|c}
x & f(x) & f(-x) & -f(x)\\\hline
\blank{1cm} & \blank{1.6cm} & \blank{1.6cm} & \blank{1.6cm}\\
\blank{1cm} & \blank{1.6cm} & \blank{1.6cm} & \blank{1.6cm}\\
\blank{1cm} & \blank{1.6cm} & \blank{1.6cm} & \blank{1.6cm}\\
\end{array}
\]

\ruleline

\section*{Reflection (3--5 min)}
\begin{itemize}
  \item What \textbf{visual cue} most reliably told you that a function matched Mirror? Spin?
  \item Which transformation most \emph{surprised} you in how it affected matching, and why?
  \item If you had to explain your findings to a friend using only pictures, how would you do it?
\end{itemize}
\vspace{1.5em}\ruleline

\section*{(Teacher) Success Criteria — Investigating Patterns}
\textit{Focus on reasoning and use of evidence.}
\begin{itemize}
  \item \textbf{Systematic exploration:} a varied set of examples/non-examples (Mirror, Spin, Neither) recorded.
  \item \textbf{Conjecturing:} clear, testable statements in students’ own words.
  \item \textbf{Testing \& refinement:} transformations and edge cases are used to probe limits.
  \item \textbf{Justification:} examples and counterexamples linked explicitly to the conjecture.
  \item \textbf{Clarity:} reasoning communicated with correct use of the applet’s observations.
\end{itemize}

\vfill
\begin{center}
\textit{Reminder: You do not submit anything online for this activity. Use the applet to explore and this sheet to capture your thinking.}
\end{center}

\end{document}
